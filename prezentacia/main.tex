\documentclass{beamer}
\usepackage{slovak}
\usepackage[utf8]{inputenc}
\usepackage{epsfig}
\usepackage{fancyhdr}
\usepackage{lastpage}
\usepackage{url}
\usepackage{color}
\usepackage{amsfonts}
\usepackage{amssymb}
\usepackage{verbatim}
\usepackage{fancybox}
\usepackage{tabularx}
\usepackage{wrapfig}
\usepackage{graphicx}
\usepackage{fancyvrb}
\usepackage{ksp-lst-utf8}

\mode<presentation>
{
  \usetheme{Warsaw}
  \usecolortheme{seahorse}
  \usecolortheme{rose}
  \setbeamercovered{transparent}
}

\title{ViDA: Vizualizácia distribuovaných algoritmov}

%\uv{Zaba}
\author{Michal Anderle \newline Ján Hozza}

\date[ŠVK]{ŠVK -- 23. Apríl 2013}

\begin{document}
\begin{frame}
  \titlepage
\end{frame}

\begin{frame}
    \frametitle{Distribuované algoritmy}

    \begin{block}{Vlastnosti}
        \begin{itemize}
        \vfill\item niekoľko počítačov zapojených do siete obojsmernými komunikačnými linkami
        \vfill\item komunikácia prebieha posielaním správ
        \vfill\item správy sa nestrácajú, ale nie je garantovaná rýchlosť posielania --
        asynchonizácia
        \end{itemize}
    \end{block}

    \begin{block}{Ciele}
        \begin{itemize}
        \vfill\item procesy sa snažia spoločne vyriešiť daný problém
        \vfill\item snaha poslať čo najmenej správ
        \vfill\item klasické problémy -- voľba šéfa, broadcast, traverzovanie \dots
        \end{itemize}
    \end{block}

\end{frame}

\begin{frame}
    \frametitle{Hlavné ciele našej aplikácie}

    \begin{block}{}
        \begin{itemize}
            \vfill\item jednoduché používanie
            \vfill\item vizualizácie konkrétnych algoritmov s priblížením ich fungovanie
            \vfill\item interaktivita s používateľom
            \vfill\item možnosť naprogramovať si vlastné vizualizácie
        \end{itemize}
    \end{block}

\end{frame}

\begin{frame}
    \frametitle{Výzor a nástroje}

    \begin{block}{}
    \begin{itemize}
        \vfill\item procesy a linky reprezentovné grafom
        \vfill\item bublinkový interface na zobrazovanie informácií
        \vfill\item grafické okno poskytujúce možnosť rýchleho editovania
        \vfill\item možnosť časovania správ
        \vfill\item neprehltenosť dátami -- menej podstatné informácie sa dajú zobrazovať dodatočne
    \end{itemize}
    \end{block}

    %TODO obrazok

\end{frame}

\begin{frame}
    \frametitle{Vizualizované algoritmy}

    \begin{block}{}
    \begin{itemize}
        \vfill\item voľba šéfa na úplnom grafe s $O(n\log n)$ správami
        \vfill\item distribuované BFS -- broadcast
        \vfill\item distribuované DFS -- traverzovanie
    \end{itemize}
    \end{block}

\end{frame}

\begin{frame}
    \frametitle{Tvorba vlastných vizualizácií}

    \begin{block}{}
    \begin{itemize}
        \vfill\item jednoduchá C++ knižnica vidalib
        \vfill\item prístup ku všetkým vizualizačným nástrojom
        \vfill\item možnosť nazrieť "pod pokrývku" jednotlivým algoritmom
    \end{itemize}
    \end{block}

    %TODO obrazok

\end{frame}

%\listing{code/stack.cpp}
%\includegraphics[height=0.5\textheight]{img/tree.png}

\end{document}
