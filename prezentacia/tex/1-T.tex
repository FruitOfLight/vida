\nadpis{T}

Žaba si chce zarobiť na brigádach. Preto si vyhliadol, aké
rôzne brigády sa konajú nasledujúce dni. Je ich $n$. Každá
má daný čas kedy začína, kedy končí a koľko peňazí na nej
zarobím. Zistite koľko najviac peňazí vie Žaba zarobiť, ak
peniaze dostane len ak sa zúčastní celej brigády a ak nejaká
končí vtedy keď iná začína, vie stihnúť obe.

\vstup
3
1 3 2
3 8 4
2 4 10
\vystup
10
\komentar
Zúčastní sa len jednej brigády v čase 2-4, lebo je dobre platená.
\koniec

Riešenie: Usortím si brigády podľa času konca. Snažím sa robiť dynamiku:
koľko viem zarobiť za prvých $i$ brigád. To zistím tak, že zoberiem maximum
z $D[i-1]$ a $c(i)+D[j]$ kde $j$
je taká brigáda, čo končí skôr ako $i$-ta
začína. Hodnotu $j$ binárne vyhľadávam.
