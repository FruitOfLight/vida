\nadpis{T. Križovatky ( 9 ) }

Máme cestnú sieť, ktorú tvoria križovatky a cesty medzi nimi. 

Zaujíma nás, koľko áut vie prejsť z
križovatky $A$, do križovatky $B$, ak každou križovatkou okrem $A$ a $B$ môže pprejsť najviac jedno auto.

Aby to nebolo také zložité, už máme funkciu, ktorej dáme ľubovoľný neohodnotený graf $G$ a dva jeho vrcholy
$u$,$v$ a funkcia nám povie, aký je maximálny tok medzi $u$ a $v$. 

\nadpis{vzorák}
Vrchol $x$ rozdelíme na $x1$ a $x2$, pričom do $x1$ pôjdu orientované hrany, ktoré šli do $x$, z $x1$
do $x2$ pôjde jedna hrana s kapacitou 1 a z $x2$ pôjdu von orientované hrany, ktoré šli z $x$.
Podhodíme tokovej funkcii.

