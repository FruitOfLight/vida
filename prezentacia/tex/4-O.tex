\nadpis{O}

Sultán má šnúru $n$ perál, ktoré ležia
na stole tak, že práve $m$ z nich visí dole
zo stola. Jeho sluha mu chce zopár perál ukradnúť,
ale musí si dávať pozor. Každý deň vie zobrať perlu
len z jedného jej konca, ale vždy musí práve $m$
perál visieť dolu zo stola. Samozrejme perly majú
svoje váhy. Označme $V_d$ váhu visiacich perál
a váhu $V_n$ nevisiacich. Ak platí $V_d>k.V_n$ pre dané
$k$ perly spadnú na zem, čo sa nemôže stať.
A samozrejme každá perla má svoju cenu, kažá inú
a sluha chce zarobiť. Poraďte sluhovi ako má perly
kradnúť aby nakradol čo najviac a nebol spozorovaný.

\vstup
5 2 1
5 3
4 2
6 4
3 2
2 2
\vystup
2 5
NN
\komentar
môže zobrať najviac dve z Nevisicej časti.
\koniec

Riešenie: Spravím nejaké prefixové sumy. Vždy sa oplatí
najskôr brať zo spodu a až potom zvrchu. Pozriem všetky
možnosti čo zoberiem zo spodu a druhú hranicu binárne dohľadám
(alebo aj nie)
