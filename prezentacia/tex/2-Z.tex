\nadpis{Z. Kartograf ( 2 + 3 ) }
Máme zadanú maticu rozmerov $r \times s$  na ktorej sú znaky '.' a '\#'. '.' je voda a '\#' je pevnina.
Ostrov je súvislý kus pevniny - t.j. medzi každými dvoma kúskami ostrova sa dá dostať tak, že sa
pohybujeme len hore, dole, doľava a doprava a nikdy nevstúpime do vody.

Máte zaručené, že celý okraj mapy je voda.

Zaujíma nás:

a) počet všetkých ostrovov

b) najväčší z ich obvodov

\vstup
5 9
.........
.\#..\#\#.\#.
...\#\#....
.....\#\#..
.........
\vystup
4
10
\koniec

\nadpis{vzorák}
Prejdeme celé pole a vždy keď nájdeme neofarbenú pevninu, ofarbíme ostrov,
napríklad pomocou dfs. V dfs si počítam, koľkokrát som mal za suseda 
vodu a to bude obvod ostrova.

T: $O(rs)$; S: $O(rs)$.

