\nadpis{T. Konzervy ( 10 ) }

Babka v pivnici skladuje konzervy s ovocím. Má ich tam poukladané pekne jednu vedľa druhej v
pravidelnej mriežke $n \times n$ a každá konzerva má nejaký objem.

Chceme z pivnice uchmatnúť niekoľko konzierv, ktoré majú dokopy čo najväčší objem a zároveň chceme,
aby si babka nič nevšimla - teda aby z každého štvorca rozmerov $2 \times 2$ chýbala najviac jedna
konzerva. 

Aký najvačší objem vieme dosiahnuť?

\vstup
n = 3
2 1 2
1 4 3
3 2 1

\vystup
8

\komentar
Napríklad môžeme zobrať konzervy na pozíciách $(1,1)$, $(2,3)$, $(3,1)$. $2+3+3 = 8$

\koniec

\nadpis{vzorák}
Klasicka dynamika, $n\cdot \phi^{2n}$.

Uplný vzorák, idem po riadkoch po políčkach a pamätám si všetky možnosti pre posledných n+1 bitov.
Z nich viem rýchlo vyberať správne, keď sa rozhodujem, či zoberiem alebo nezoeriem nejaké konktrétne
políčko. To sa dá robiť v čase $n^2 \cdot \phi^{n+1}$.

T: $O(n^2 \cdot \phi^n)$; S: $O(\phi^n)$.
