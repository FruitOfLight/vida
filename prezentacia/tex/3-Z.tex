\nadpis{Z. ( Prorok na koni (5) ) }

Majme šachovnicu 9 krát 9, v dolnom pravom rohu máme špeciálnu šachovú figárku, 
ktorú nazveme \emph{Prorok na koni}. 
\emph{Prorok na koni}
vie spraviť skok na vedľajšie políčko (4 možnosti, ak nie je na okraji alebo
v rohu), a skok ako klasický kôň do L (8 možností, ak neskočí mimo šachovnicu). 

Chceli by sme spraviť postupnosť skokov tak, aby \emph{Prorok na koni} navštívil
každé políčko práve raz a skončili na políčku, na ktorom začal. 
Dá sa to? Ak áno, nájdite postupnosť skokov, ak nie, dokážte.

\nadpis{vzorák}
Nedá sa to, no pre spor predpokladajme, že sa to dá. 

\emph{Prorok na koni} začne na políčku nejakej farby, BUNV na bielej. Po každom skoku sa farba políčka
pod ním zmení. Musí spraviť 81 skokov, takže skončí na čietnom políčku. To je ale spor s tým, že
skončil na tom políčku, kde začal.


