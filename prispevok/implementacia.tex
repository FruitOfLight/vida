\section{Implementácia}

Projekt ViDA vznikol ako ročníkový projekt a nevychádzal zo žiadnych predošlých prác alebo
projektov. Celý výzor sme navrhli sami po konzultácii s našim vedúcim. Ako to vyzerá, môžete vidieť
na prvom obrázku. Naším prvoradým cieľom, bolo zabezpečiť prehľadnú vizualizáciu a tiež jednoduché
používanie aplikácie.

\subsection{Vizualizácia}

Pri distribuovaných algoritmoch sa bavíme o sieti počítačov, ktoré medzi sebou komunikujú --
posielajú si medzi sebou správy. Preto bolo prirodzené, aby sme si reprezentovali počítače a ich
vzájomné zapojenie do siete ako graf. Každý vrchol je jeden počítač (proces) a správy sa presúvajú
medzi počítačmi, len po hranách v danom grafe.

Keďže samotná sieť počítačov má veľký vplyv na vykonávanie daného algoritmu, v prvej fáze, bolo
treba umožniť užívateľovi si pohodlným spôsobom vytvárať a editovať graf. Na to je určená hlavná
plocha, kde v dobe keď nebeží žiaden algoritmus, môže pomocou myši meniť výzor grafu. Na spodnej
lište je niekoľko tlačidiel, ktoré umožňujú užívateľovi prepínať si medzi možnosťami pridávanie,
odstraňovanie alebo pohybovanie. S vybratým módom, môže užívateľ vykonávať danú operáciu na grafe.
Takisto, existuje pár známych a často používaných typov grafov, ktoré si úživateľ môže zvoliť priamo
aj s danou veľkosťou grafu.

Samozrejme najdôležitejšia časť je vizualizácia samotného algoritmu. Bolo dôležité aby to bolo čo
najprehľadnejšie a zároveň to dávalo dostatok informácií. Z viacerých možností, ktoré sme skúšali
sme nakoniec vybrali bublinkový interface, kde všetky podstatné informácie sú zobrazované vo
vyskakovacích bublinkách.

Počas algoritmu jednotlivé procesy často oznamujú nejakú informáciu, aby dali najavo, čo sa v nich
deje. Zo začiatku sme tieto informácie zobrazovali v okienku napravo, aby sme nezahltili priestor
grafu zbytočnými textami. Toto všeobecné okienko bolo však mimo grafu a bolo skoro nemožné sledovať
čo sa deje vnútri grafu (kde ide aká správa) a zároveň si dávať pozor, čo vraví ktorý vrchol. Preto
sme sa rozhodli, že informácie vrchola sa budú zobrazovať priamo pri vrchole vnútri grafu, čo sa
nakoniec ukázalo, že nie je až tak zavadzajúce, hlavne ak tieto správy časom miznú. Ponechali sme aj
panel naboku, ktorý má slúžiť, keď si chce užívateľ pozrieť, čo sa dialo v histórii. A tento panel
nezavadzá, keďže je decentne skrytý a ukáže sa až na užívateľov príkaz.

Pri distribuovaných algoritmoch nás zaujíma hlavne to, ako sa algoritmus presúva z jedného stavu do
druhého na základe doručenej správy. Pod pojmom stav rozumiem naplnenie niektorých dôležitých
premenných. Je teda zjavné, že je dôležité, aby úživateľ videl (alebo aspoň tušil), akú hodnotu majú
tieto dôležité premenné. Dôležité premenné sú napríklad ID vrchola, či je živý alebo nie, jeho
level \dots Vypisovať tieto premenné v bublinkách pri vrchole, by však bolo zavadzajúce, keďže je to
informácia, ktorá je potrebná stále. Preto sme zvolili taký prístup, aby výzor vrcholu reprezentoval
dané premenné.

Keďže ID vrchola je jedna z najdôležitejších informácii, lebo je nezávislá od algoritmu, ktorý je
spustený a teda často slúži na prelomenie určitej symetrie, táto hodnota sa vypisuje priamo vnútri
vrchola. Ďalšie atribúty vrcholu sú jeho farba, alebo veľkosť. Preto naše vizualizácie často
využívajú tieto vlastnosti a intuitívne ich spájajú s nejakou premenou. Napríklad mŕtvy proces zmení
svoju farbu na červenú, alebo zväčšujúci sa level zväčšuje veľkosť vrchola.

Samozrejme, občas je dôležité aby sa užívateľ mohol pozrieť na skutočnú hodnotu danej premennej.
Preto si vie označiť vrchol, ktorý mu vo vedľajšom okienku ukáže premenné daného procesu.
