\section{Úvod}

V zimnom semestri sme absolvovali predmet Úvod do distribuovaných algoritmov a veľmi nás zaujala
prednášaná téma. Napadlo nás, že táto téma, aj keď je zložitá, by sa dala vysvetliť bežnému človeku,
ktorý má o ňu záujem bez toho, aby si musel študovať zložito písané knihy. Tak vznikol náš ročníkový
projekt.

Ako cieľ sme si stanovili nielen vizualizovanie niektorých dobre známych algoritmov, ale taktiež
vytvorenie istej interaktivity s používateľom, ktorá by mu pomohla prehĺbiť a lepšie si zapamätať
novonadobudnuté znalosti. Myslíme si totiž, že v tejto oblasti existuje mnoho pekných úvah a trikov,
ktoré je dobré si osvojiť. A formou interaktívnej vizualizácie, by sme chceli pomôcť vo
výučbe a spoznávaní týchto algoritmov, ale takisto umožniť užívateľovi, aby si sám vyskúšal
naprogramovať niektoré z algoritmov. Toto by malo preklenúť medzeru medzi porozumením a schopnosťou
aplikácie.

Medzi algoritmy, ktoré sme zatiaľ zvizualizovali, patrí distribuované prehľadávanie do šírky,
taktiež voľba šéfa na úplnom grafe a traverzovanie. Pri vývoji aplikácie sme dbali aj na to, aby sa dali jednoducho
pridávať nové algoritmy, dokonca aby užívateľ bol schopný sám vytvoriť algoritmus, 
ktorý mu naša aplikácia zvizualizuje.

Zvyšok článku je organizovaný nasledovne: v sekcii $2$ si povieme, v akom výpočtovom modeli
pracujeme a fungujú spomínané tri distribuované algoritmy. 
Následne v sekcii $3$ popíšeme samotnú implementáciu aplikácie, ako
funguje a prečo sme veci robili tak, ako sme ich robili. Nakoniec v sekcii $4$ spomenieme naše plány do budúcnosti.
