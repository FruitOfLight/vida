\section{Úvod}

V zimnom semestri sme absolvovali predmet Úvod do distribuovaných algritmov a veľmi nás zaujala
prednášaná téma. Napadlo nás, že táto téma, aj keď je zložitá, by sa dala vysvetliť bežnému človeku,
ktorý má o ňu záujem bez toho, aby si musel študovať zložito písané knihy. Tak vznikol náš ročníkový
projekt.

Ako cieľ sme si stanovili nielen zvizualizovanie niektorých dobre známych algoritmov, ale taktiež
vytvorenie istej interaktivity s používatľom, ktorá by mu pomohla prehĺbiť a lepšie si zapamätať
novonadobudnuté znalosti. Myslíme si totiž, že v tejto oblasti existuje mnoho pekných úvah a trikov,
ktoré je dobré si osvojiť. A formou interaktívnej vizualizácie, by sme chceli pomôcť vo
výučbe a spoznávaní týchto algoritmov, ale takisto umožniť užívateľovi, aby si sám vyskúšal
naprogramovať niektoré z algoritmov. Toto by malo preklenúť medzeru medzi porozumením a schopnosťou
aplikácie.

Medzi algoritmy, ktoré sme zatiaľ zvizualizovali patrí distribuované prehľadávanie do šírky a
taktiež voľba šéfa na úplnom grafe. Zamerali sme sa však na to, aby užívateľ bol schopný sám
vytvoriť algoritmus, ktorý mu naša aplikácia zvizualizuje.

Zvyšok článku je organizovaný nasledovne: v sekcii $2$ sa zameriame na samotnú implementáciu, ako
funguje a prečo sme zvolili zrovna túto možnosť. V sekcii $3$ popíšeme konkrétne algoritmy, ktoré
vizualizujeme a v sekcii $4$ spomenieme naše plány do budúcnosti.
