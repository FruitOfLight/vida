\section{Plány do budúcnosti}

V budúcnosti by sme sa chceli naďalej venovať rozvoju našej aplikácie. Najdôležitejšie bude pridať
ďalšie algoritmy a modely výpočtu. Medzi modely by sme chceli zaradiť model routingu, kde si procesy
medzi sebou posielajú správy, pričom chcú komunikovať aj iné ako susedné procesy. Ďalšie modely sú
založené na tom, že sieť sa môže kaziť, teda sa môže stratiť správa alebo dokonca havarovať celý
proces.

Medzi algoritmy by sme chceli zaradiť, voľbu šéfa na kruhu, voľbu šéfa na ľubovoľnom grafe
(GHS\footnote{G. Tel: Introduction to distributed algorithms, ( Chapter 7 : Election Algorithms ),
Cambridge University Press, 1994,2000}) ale
tiež routingový Netchange, či rôzne typy kompaktných routovaní. Ďalej algoritmy dohody odolné voči
chybám či iné spôsoby traverzovania.

Tiež by sme chceli prehĺbiť interaktivitu s používateľom. Chceme, aby mohol viac ovplyvniť dianie
konkrétneho algoritmu za pomoci našich nápovedných rád a taktiež mu dávať rôzne výzvy štýlu: "Donúť
tento algoritmus poslať aspoň $100$ správ.". Toto by malo pomôcť hlbšiemu pochopeniu daného
algoritmu.

Rovnako chceme dorobiť podporu rôznych programovacích jazykov, pre užívateľské programy a tiež
rozšírenie našej knižnice o ďalšie funkcie, aby užívateľ mohol vytvoriť plnohodnotnú vizualizáciu
svojho algoritmu.

V neposlednom rade je dôležitý výzor samotnej aplikácie a jej jednoduchosť na používanie. Je
dôležité aby bola intuitívna na ovládanie a neodstrašovala svojím výzorom. Ak sa totiž niekam nedá
dostať na pár kliknutí myšou, ako by to neexistovalo.
